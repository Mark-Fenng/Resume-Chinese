%%%%%%%%%%%%%%%%%%%%%%%%%%%%%%%%%%%%%%%%%
% Freeman Curriculum Vitae
% XeLaTeX Template
% Version 2.0 (19/3/2018)
%
% This template originates from:
% http://www.LaTeXTemplates.com
%
% Authors:
% Vel (vel@LaTeXTemplates.com)
% Alessandro Plasmati
%
% License:
% CC BY-NC-SA 3.0 (http://creativecommons.org/licenses/by-nc-sa/3.0/)
%
%!TEX program = xelatex
% NOTICE: This template must be compiled with XeLaTeX, the line above should
% ensure this happens automatically but if it doesn't you will need to specify 
% XeLaTeX as the engine in your editor or script
% 
%%%%%%%%%%%%%%%%%%%%%%%%%%%%%%%%%%%%%%%%%

%----------------------------------------------------------------------------------------
%	PACKAGES AND OTHER DOCUMENT CONFIGURATIONS
%----------------------------------------------------------------------------------------

\documentclass[10pt]{article} % Font size, can be: 10pt, 11pt or 12pt

\usepackage[UTF8]{ctex} % Chinese support

\input{structure.tex} % Include the file that specifies the document structure

% Headers and footers can be added with the \lhead{} \rhead{} \lfoot{} \rfoot{} commands
% Example right footer:
%\rfoot{\color{headings}{\sffamily Last update: \today. Typeset with Xe\LaTeX}}

%----------------------------------------------------------------------------------------

\begin{document}

\begin{paracol}{2} % Begin the multi-column environment

%----------------------------------------------------------------------------------------
%	NAME AND BASIC INFO
%----------------------------------------------------------------------------------------

\parbox[top][0.12\textheight][c]{\linewidth}{ % Parbox to hold the author name and CV text; fixed height to match the coloured box to the right, centred vertically and full line width
	\vspace{-0.04\textheight} % Reduce whitespace above the parbox to separate it from the main content
	\centering % Centre text
	{\sffamily\Huge 冯运}\\
	\vspace{0.04\textheight}
	\begin{tabular}{ccc}
		\textbf{应聘岗位} & \textbf{本科院校} & \textbf{毕业时间}\\
		\hline
		Web 前端 & 哈尔滨工业大学\,计算机学院  & 2020.07\\
	\end{tabular}
}

%----------------------------------------------------------------------------------------
%	MAJOR RESEARCH PROJECT
%----------------------------------------------------------------------------------------

\section{个人简介}

本科就读于哈尔滨工业大学计算机科学与技术学院,已于2020年7月毕业。原本计划去美国继续研究生学业,但是由于疫情原因无法前往,我打算先工作来积累工作经验。我喜欢编程,擅长动手实践,大学的大部分业余时间都用来完成自己的项目,参加竞赛以及在实验室参与研究。在本科课程学习中,我自己动手实现了一个基于哈夫曼编码的文件压缩工具,基于Verilog的五级流水线Y86指令集和基于滑动窗口的可靠传输协议等。在实验室的研究经历也有不少收获,目前还有一篇关于图数据库的一作论文已被NDBC(2020)会议录用。
\medskip % Extra whitespace before the next section

%----------------------------------------------------------------------------------------
%	EDUCATION
%----------------------------------------------------------------------------------------

\section{教育经历} 

% Blank \educationentry{} command to add another degree:

%\educationentry{} % Duration
%{} % Degree
%{} % Honours, achievements or distinctions (e.g. first class honours)
%{} % Department
%{} % Institution

% All 5 parameters must be supplied but any can be empty if you don't need them

%------------------------------------------------

\begin{supertabular}{rl} % Start a table with two columns, the table will ensure everything is aligned

	%------------------------------------------------
	
	\educationentry{2016年8月 -- 2020年7月} % Duration
	{工学学士} % Degree
	{} % Honours, achievements or distinctions (e.g. first class honours)
	{计算机科学与技术专业} % Department
	{哈尔滨工业大学} % Institution
	
	%------------------------------------------------

\end{supertabular}

%----------------------------------------------------------------------------------------
%	AWARDS
%----------------------------------------------------------------------------------------

\section{获奖情况}

% Example \tableentry{} command to add another line:

%\tableentry{Heading}{Content}{spaceafter}

% All 3 parameters must be supplied but any can be empty if you don't need them
% A "spaceafter" value in the third parameter will add some vertical space -- this is to be used between headings

%------------------------------------------------

\begin{supertabular}{rl} % Start a table with two columns, the table will ensure everything is aligned
	
	%------------------------------------------------
	
	\tableentry{2019年7月}{\textbf{2019微信小程序应用开发赛全国三等奖}}{spaceafter}
	\tableentry{}{\textit{微信小程序大赛组委会}}{spaceafter}
	
	%------------------------------------------------

	\tableentry{2019年3月}{\small\textbf{第十届蓝桥杯大赛javaA组黑龙江赛区二等奖}}{spaceafter}
	\tableentry{}{\textit{蓝桥杯大赛组委会}}{spaceafter}
	
	%------------------------------------------------

	% \tableentry{2018年12月}{\small\textbf{全国大学生数学建模竞赛黑龙江赛区一等奖}}{spaceafter}
	% \tableentry{}{\textit{黑龙江赛区组委会}}{spaceafter}
	
	% %------------------------------------------------
	
\end{supertabular}

%----------------------------------------------------------------------------------------
%	COMPUTER SKILLS
%----------------------------------------------------------------------------------------

\section{技术能力} 

% Example \tableentry{} command to add another line:

%\tableentry{Heading}{Content}{spaceafter}

% All 3 parameters must be supplied but any can be empty if you don't need them
% A "spaceafter" value in the third parameter will add some vertical space -- this is to be used between headings

%------------------------------------------------

\begin{supertabular}{rl} % Start a table with two columns, the table will ensure everything is aligned
	
	%------------------------------------------------
	
	\tableentry{熟练}{web前端基础}{spaceafter}
	\tableentry{}{Vue框架}{spaceafter}
	\tableentry{}{前端工程化有较多实践}{spaceafter}	
	
	%------------------------------------------------
	
	\tableentry{熟悉}{java基础}{spaceafter}	
	\tableentry{}{Linux系统(日常生活使用Ubuntu)}{spaceafter}
	
	%------------------------------------------------
	
\end{supertabular}

%----------------------------------------------------------------------------------------
%	COMMUNICATION SKILLS
%----------------------------------------------------------------------------------------

\section{语言能力}

% Example \tableentry{} command to add another line:

%\tableentry{Heading}{Content}{spaceafter}

% All 3 parameters must be supplied but any can be empty if you don't need them
% A "spaceafter" value in the third parameter will add some vertical space -- this is to be used between headings

%------------------------------------------------

\begin{supertabular}{rl} % Start a table with two columns, the table will ensure everything is aligned
	
	%------------------------------------------------
	
	\tableentry{CET-6}{539分}{}
	
	%------------------------------------------------
	
	\tableentry{托福}{102分}{}
	
	%------------------------------------------------
	
\end{supertabular}

%----------------------------------------------------------------------------------------

\switchcolumn % Switch to the next paracol column

%----------------------------------------------------------------------------------------
%	COLOURED CONTACT DETAILS BOX
%----------------------------------------------------------------------------------------

\parbox[top][0.12\textheight][c]{\linewidth}{ % Parbox to hold the colour box; fixed height to match the name/CV text to the left, centred vertically and full line width
	\vspace{-0.04\textheight} % Reduce whitespace above the parbox to separate it from the main content
	\colorbox{shade}{ % Create the coloured box
		\begin{supertabular}{p{0.05\linewidth}|p{0.775\linewidth}} % Start a table with two columns, the table will ensure everything is aligned
			\raisebox{-1pt}{\faHome} & \small{山西省吕梁市离石区团结路195号} \\ % Home address			
			\raisebox{-1pt}{\faPhone} & +86 16601210817 \\ % Phone number
			\raisebox{0pt}{\small\faEnvelope} & \href{mailto:fengyun5264@outlook.com}{fengyun5264@outlook.com} \\ % Email address
			\raisebox{-1pt}{\faGithub} & \href{https://github.com/Mark-Fenng}{https://github.com/Mark-Fenng} \\ % GitHub profile
		\end{supertabular}
	}
}

%----------------------------------------------------------------------------------------
%	WORK EXPERIENCE
%----------------------------------------------------------------------------------------

\section{项目经历}

% Blank \projectentry command to add another project:

%------------------------------------------------
% Project 1
{\large{\textbf{侧边翻译}}} \hfill \textsc{2018年7月 -- 至今} % Project name and duration
	\begin{itemize}
		\item \textbf{项目主页:}\href{https://github.com/EdgeTranslate/EdgeTranslate}{\small{https://github.com/EdgeTranslate/EdgeTranslate}} % Project home page
		\item \textbf{项目简介:}{这是我与室友合作的一个\textbf{浏览器翻译扩展}项目,它提供便捷高效的网页和PDF翻译功能。这个项目一发布就在Github上开源,到目前为止已经拥有了\textbf{100K+}的周活用户,
		在Github上获得了\textbf{737个}Star,在Chrome、Firefox和Edge上均获得\textbf{4.7分(满分5分)}以上的评分。} % Project description
		\item \textbf{主要职责:}{参与了扩展从一个想法到成熟软件的全过程,包括功能的讨论、交互方式设计、页面样式设计、功能实现、发布和推广等。在代码上,我实现了翻译结果的展示模块,划词翻译的功能,右上角弹出框和扩展设置页面等部分,同时也负责扩展的长期维护工作。} % Duty in the project
		\item \textbf{项目优点:}{
			\begin{itemize}			
			\item 从零开始自己动手搭建了完善的开发环境,通过简单的命令就能完成开发,打包到不同浏览器的工作,包括热更新和单元测试环境;
			\item 自己实现了一个能拖动和调整元素大小的库
			\item 自己实现了一个轻量的模板引擎,用于渲染翻译结果页面			
			\item 功能实用,交互方式高效,用户量增长迅速,广受好评,长期维护;
			\item 完善的文档,大量代码注释,详尽的release发布记录
		\end{itemize}} % Project highlights
	\end{itemize}

%------------------------------------------------

\projectentry
{教育网站开发}
{2017年11月 -- 2018年5月}
{https://github.com/Mark-Fenng/QingXueFrontEnd}
{\small {https://github.com/Mark-Fenng/QingXueFrontEnd}}
{这个项目是在\textbf{大学二年级}时抽业余时间做的一个外包项目,做的是一个销售教育课程的网站,包括用户系统和教育机构系统两部分,总共15个页面,使用的技术包括\textbf{Vue系列组件}和Element UI。在项目中,我同时参与了大量功能上的讨论和后端数据库设计,还负责了网站在阿里云的部署维护工作。} % Description
{这个项目的前端部分页面设计和代码均为我独自完成。} % Duty

%------------------------------------------------

% \projectentry
% {My Pregel}
% {2019年6月}
% {https://github.com/nickyc975/Pregel}
% {https://github.com/nickyc975/Pregel}
% {这个项目是数据挖掘课程的课设。\textbf{Pregel}是Google提出的一个\textbf{分布式并行图计算系统},其核心是
% \textbf{基于BSP模型的分布式计算框架},更详细的信息可以参考这篇论文:\href{https://www.cs.cmu.edu/~pavlo/courses/fall2013/static/papers/p135-malewicz.pdf}{Pregel: a system for large-scale graph processing}。
% 本项目是Pregel计算模型的单机版实现,使用了\textbf{多线程}来实现并行计算。实现了\textbf{Combiner},
% \textbf{Aggregator}等Pregel核心功能,同时提供了\textbf{函数式API},使用起来简洁方便。} % Description
% {这个项目是一个完全的个人项目,所有部分均为我独自实现。} % Duty

%----------------------------------------------------------------------------------------

\end{paracol}

%----------------------------------------------------------------------------------------

\end{document}
